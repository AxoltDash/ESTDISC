\documentclass[a4paper,10pt]{article}
\usepackage{bookmark}
\usepackage[utf8]{inputenc}
\usepackage[spanish]{babel}
\usepackage{hyperref}
\usepackage{amsfonts}
%Multicolumns
\usepackage{multicol}
\usepackage{fdsymbol}
\usepackage{tabularx}
\usepackage{graphicx}
\usepackage{float}
\usepackage{booktabs}
\usepackage{marginnote}
\usepackage{enumitem}
\usepackage{lineno}
\pagenumbering{gobble}


\newcolumntype{A}{ >{$} r <{$} @{} >{${}} l <{$} } 


\title{Estructuras Discretas\\Tarea 9 \textbf{Por: P\'erez Servin Darshan Israel}\\ Fecha de entrega: martes 21 de noviembre de 2023}
\author{Profesor: Nestaly Mar\'in Nev\'arez \\ Ayudantes de teor\'ia: Eduardo Pereyra Zamudio  \\ \phantom{Ayudantes de teor\'ia:} Ricardo L\'opez Villaf\'an \phantom{aa}\\ Ayudantes de laboratorio: Edgar Mendoza Le\'on \phantom{aaai}\\ \phantom{Ayudantes de laboratorio:} David Valencia Rodr\'iguez}
\date{}

\begin{document}

\maketitle

\vspace{-15pt}
Resuelva de manera limpia y ordenada los siguientes ejercicios. 
Indique claramente el n\'umero de pregunta que se esta resolviendo.


\begin{enumerate}

	\item \marginnote{\it2 puntos} Traduzca los siguientes enunciados a l\'ogica de predicados. 
  Indique de manera clara el universo de discurso, los predicados que utilizar\'a, y a qu\'e inciso corresponde cada f\'ormula. Defina un \'unico universo de discurso para todos los predicados.\\
    \\ \textbf{Universo:} Criaturas
    \\ \textbf{D(x):} x es un drag\'on
    \\ \textbf{G(x):} x es un grifo
    \\ \textbf{M(x,y):} x es m\'as fuerte que y
  \begin{enumerate}    
    \item Existen a lo m\'as dos dragones.
    $$\exists x \exists y (D(x) \wedge D(y) \wedge \neg \exists z (D(z) \wedge z \neq x \wedge z \neq y))$$
    \item Existen exactamente dos dragones.
    $$\exists x \exists y (D(x) \wedge D(y) \wedge z \neq y \wedge \neg \exists z (D(z) \wedge z \neq x \wedge z \neq y))$$
    \item S\'olo existe un grifo que es m\'as fuerte que todos los dragones.
    $$\exists x (G(x) \wedge \forall y(D(x) \to M(x,y)))$$
    \item Para todo drag\'on hay un grifo que es m\'as fuerte que \'el.
    $$\forall x(D(x) \to \exists y ( G(g) \wedge M(g, d)))$$
  \end{enumerate}
  
 \item \marginnote{\it4 puntos} Considere los siguientes predicados:
  \begin{itemize}
    \item $P(x)$ $x$ es un n\'umero par.
    \item $M(x,y)$ $x$ es menor que $y$.
    \item $D(x,y)$ la divisi\'on de $x$ entre $y$ est\'a dentro del conjunto.
  \end{itemize}

  (Por ejemplo, la divisi\'on de los n\'umeros naturales $6$ entre $2$ es $3$, as\'i que $D(6,2)$ es verdadero en los n\'umeros naturales. La divisi\'on de los n\'umeros enteros $-5$ entre $2$ es $-5/2$, por lo que $D(-5,2)$ es falso en los n\'umeros enteros, mientras que en el dominio de los n\'umeros reales es verdadero.)
  Y los siguientes enunciados:\\\\
  \textbf{"Cabe aclarar que los contraejemplos no estan estrictamente relacionados con los enunciados, osea que no son directamente una negación de los enunciados y solo se usó lenguaje lógico para explicar los contraejemplos."}\\
  \begin{enumerate}[label=\arabic*)]
    \item $\forall x \forall y \left(M\left(x,y\right) \to \exists z\left(M\left(x,z\right) \wedge M\left(z,y\right)\right)\right)$
    \begin{enumerate}[label=\alph*)]
      \item \textbf{Es falso para los naturales.}\\
      Entonces: $\exists x \exists y (((x = 2) \wedge (y = 3)) \to \neg \exists z (M(x,z) \wedge M (z,y)))$
      \item \textbf{Es falso para los enteros.}\\
      Entonces: $\exists x \exists y (((x = -5) \wedge (y = -4)) \to \neg \exists z (M(x,z) \wedge M (z,y)))$
      \item \textbf{Es verdadero para los reales.}\\
    \end{enumerate}

    \item $\forall x \left(\left(P\left(x\right) \wedge x \neq 0\right) \to M\left(0,x\right)\right)$
    \begin{enumerate}[label=\alph*)]
      \item \textbf{Es verdadero para los naturales.}
      \item \textbf{Es falso para los enteros.}\\
      Entonces: $\exists x ((x = -2) \to \neg M(0,x))$
      \item \textbf{Es falso para los reales.}\\
      Entonces: $\exists x ((x = -4.6) \to \neg M(0,x))$\\
    \end{enumerate}

    \item $\forall x \forall y \left(x \neq 0 \wedge y\neq 0 \to \left(D\left(x,y\right) \wedge D\left(y,x\right)\right)\right)$
    \begin{enumerate}[label=\alph*)]
      \item \textbf{Es falso para los naturales.}\\
      Entonces: $\exists x \exists y (x = 2 \wedge y = 3 \to (\neg D(x,y) \wedge \neg D(y,x)))$
      \item \textbf{Es falso para los enteros.}\\
      Entonces: $\exists x \exists y (x = -4 \wedge y = -7 \to (\neg D(x,y) \wedge \neg D(y,x)))$
      \item \textbf{Es verdadero para los reales.}\\
      
    \end{enumerate}

    \item La negaci\'on del inciso 3)\\
    $\equiv \neg (\forall x \forall y (x\neq 0 \wedge y \neq 0 \to (D(x,y) \wedge D(y,x))))$\\
    $\equiv (\exists x \exists y ((x\neq 0 \wedge y \neq 0) \lor \neg (D(x,y) \wedge D(y,x))))$\\
    
    \begin{enumerate}[label=\alph*)]
      \item \textbf{Es verdadero para los naturales.}
      \item \textbf{Es verdadero para los enteros.}
      \item \textbf{Es falso para los reales.}\\
      Entonces:\\
      $(\exists x \exists y ((x = 1 \wedge y = 3) \to ((x\neq 0 \wedge y \neq 0) \lor \neg (D(x,y) \wedge D(y,x)))))$\\
      $(\exists x \exists y ((x = 1 \wedge y = 3) \to ((false) \lor \neg (true))))$\\
      $(\exists x \exists y ((x = 1 \wedge y = 3) \to ((false) \lor (false))))$\\
      $(\exists x \exists y ((x = 1 \wedge y = 3) \to (false)))$\\

    \end{enumerate}

  \end{enumerate}

  Eval\'ue el valor de verdad de cada enunciado con respecto a cada uno de los siguientes universos de discurso. Para aquellos enunciados que sean falsos, exhiba un contraejemplo (una asignaci\'on de individuos del universo en cuesti\'on a las variables tales que el enunciado sea falso).
  \begin{enumerate}
    \item Los n\'umeros naturales (incluyendo el 0).
    \item Los n\'umeros enteros.
    \item Los n\'umeros reales.
  \end{enumerate}
    
    \item \marginnote{\it4 puntos} Demuestre utilizando inducci\'on que las siguientes f\'ormulas se cumplen.			
	    \begin{enumerate}
			  \item $\sum\limits_{k=1}^n k\,(k!) = (n+1)!-1$\\
        Definición de inducción sobre $n$:
        \begin{itemize}
          \item Caso $n = 0$
          \item P.D $\sum\limits_{k=1}^0 k\,(k!) = (0+1)!-1$
          \item Demostramos lado derecho de la fórmula primero:
			    $\sum\limits_{k=1}^0 k\,(k!) = 0$ \hfill Por (LO SABESMOS GRACIAS A LA CLASE PERO NO ME ACUERDO DEL NAME)
          \item Ahora demostramos lado derecho de la fórmula:
			    $(0+1)!-1 = (1)!-1$ \hfill Por inducción matemática
			    $(0+1)!-1 = (1)!-1$ \hfill Por inducción matemática
      \end{itemize}
				\item $\sum\limits_{k=1}^n (2k - 1)^3 = n^2(2n^2 - 1)$				
			\end{enumerate}	

	\end{enumerate}	
%\linenumbers

\end{document}
